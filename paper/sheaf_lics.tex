\documentclass[conference]{IEEEtran}

\usepackage[utf8]{inputenc}
\usepackage[T1]{fontenc}
\usepackage{xspace}
\usepackage[colorlinks=true]{hyperref}

% *** MATH PACKAGES ***
%
\usepackage[cmex10]{amsmath}
\usepackage{amssymb,amsthm,amsfonts}


\newtheorem{thm}{Theorem}
\newtheorem{cor}[thm]{Corollary}
\newtheorem{prop}[thm]{Proposition}
\newtheorem{defi}[thm]{Definition}
\newtheorem{lem}[thm]{Lemma}
\newtheorem{ax}[thm]{Axiom}

\newcommand{\mynote}[2]{
    \fbox{\bfseries\sffamily\scriptsize#1}
    {\small$\blacktriangleright$\textsf{\emph{#2}}$\blacktriangleleft$}~}

\newcommand\kq[1]{\mynote{KQ}{#1}}
\newcommand\nt[1]{\mynote{NT}{#1}}

\newcommand{\ie}{i.e,\xspace}
\newcommand{\eg}{e.g,\xspace}

% *** Do not adjust lengths that control margins, column widths, etc. ***
% *** Do not use packages that alter fonts (such as pslatex).         ***
% There should be no need to do such things with IEEEtran.cls V1.6 and later.
% (Unless specifically asked to do so by the journal or conference you plan
% to submit to, of course. )


% correct bad hyphenation here
\hyphenation{op-tical net-works semi-conduc-tor}


\usepackage{enumerate}
\usepackage{wasysym}
\usepackage[all]{xy}
\def\dar[#1]#2{\ar@<-#2>[#1]\ar@<#2>[#1]} %double arrows in xy
\def\tar[#1]#2{\ar@<#2>[#1]\ar@<0pt>[#1]\ar@<-#2>[#1]} %triple arrows in xy
\DeclareMathOperator{\Type}{Type}
\DeclareMathOperator{\HProp}{HProp}
\DeclareMathOperator{\IsHProp}{IsHProp}
\DeclareMathOperator{\nat}{nat}
\DeclareMathOperator{\Unit}{Unit}
\DeclareMathOperator{\im}{Im}
\DeclareMathOperator{\id}{id}
\DeclareMathOperator{\Contr}{Contr}
\DeclareMathOperator{\IsContr}{IsContr}

\def\mymathhyphen{{\hbox{-}}}

\newcommand{\IsType}[1]{\mathop{\mathrm{Is\mymathhyphen}#1\mathrm{\mymathhyphen
      type}} }

\newcommand{\modal}{\ensuremath{\ocircle}}
\newcommand \True {\mathrm{true}}
\newcommand \False {\mathrm{false}}
\newcommand \closure[1] {\overline{#1}}
\newcommand \Char[1] {\chi_{#1}}%{\mathrm{char}(#1)}
\newcommand \E {\mathcal{E}}
\newcommand \Hom[1] {\mathrm{Hom}_{#1}}
\newcommand \Sh[1] {\mathrm{Sh}_{#1}}

 \IEEEpubid{} 

\begin{document}
%
% paper title
% Titles are generally capitalized except for words such as a, an, and, as,
% at, but, by, for, in, nor, of, on, or, the, to and up, which are usually
% not capitalized unless they are the first or last word of the title.
% Linebreaks \\ can be used within to get better formatting as desired.
% Do not put math or special symbols in the title.
\title{Lawvere-Tierney Sheafification\\ in Homotopy Type Theory}


% author names and affiliations
% use a multiple column layout for up to three different
% affiliations
\author{\IEEEauthorblockN{Kevin Quirin and Nicolas Tabareau}
\IEEEauthorblockA{Inria \\ \'Ecole des Mines de Nantes \\ Nantes, France}}

% conference papers do not typically use \thanks and this command
% is locked out in conference mode. If really needed, such as for
% the acknowledgment of grants, issue a \IEEEoverridecommandlockouts
% after \documentclass


% make the title area
\maketitle

% As a general rule, do not put math, special symbols or citations
% in the abstract
\begin{abstract}

  Sheafification is a popular tool in topos theory which
  allows to extend the internal logic of a topos with new
  principles. One of its most famous applications is the possibility
  to transform a topos into a boolean topos using the
  dense topology, which corresponds in essence to Gödel's double
  negation translation.
  % 
  The same construction has not been considered in Martin-Löf type
  theory because of a mismatch between topos theory and type theory. This
  mismatched has been fixed recently by considering homotopy type
  theory, an extension of Martin-Löf type theory with new
  principles inspired by category theory and homotopy theory, and
  which corresponds closely to higher topoi.
  % 
  In this paper, we give a computer-checked construction of
  Lawvere-Tierney sheafification in homotopy type theory, which allows
  in particular to give a meaning to the (propositional) law of
  excluded middle inside homotopy type theory, without axiom, being
  compatible with the full type-theoretic axiom of choice.
  
\end{abstract}

% no keywords

\section{Introduction}
\label{sec:intro}

Sheafification~\cite{maclanemoerdijk} is a very powerful geometric
construction that has been initially stated in topology and has
quickly been lifted to mathematical logic.
%
In the field of topos theory, it provides a way to construct new topoi
from already existing ones, allowing logical principles---that can not
be proven to be true or false in the old topos---to be valid (or
invalid) in the new topos.
%
The most common use of sheafification is the construction a boolean
topos, thus validating the classical principle of the law of excluded
middle (EM), from an intuitionistic topos using the dense topology. 
%
An other famous application has been developed by
Cohen~\cite{cohen1966} to prove that the continuum hypothesis is
independent of the usual axioms of Zermelo-Fr\ae nkel set theory, even
in presence of the axiom of choice (AC). The initial work of Cohen
uses forcing but can be rephrased in terms of
sheafification~\cite{maclanemoerdijk}.

Even if similar questions have been considered around the Curry-Howard
isomorphism, there is no canonical way to extend a programming
language such as type theory with new logical or computational
principle while keeping consistency automatically.
%
For instance, much effort have been done to provide a computational
content to the law of excluded middle in order to define a
constructive version of classical logic. This has lead to various
calculi, with most notably the $\lambda \mu$-calculus of
Parigot~\cite{parigot1993classical}, but this line of work has not
appeared to be fruitful to define a new version of type theory with
classical principles.
%
Other works have tried to extend continuation-passing-style (CPS)
transformation to type theory, but they have been faced with the
difficulty that the CPS transformation is incompatible with (full) dependent
sums~\cite{barthe2002cps}, which puts emphasis on the tedious link
between the axiom of choice and the law of excluded middle in type theory.

Nevertheless the axiom of choice has shown to be realizable by
computational meaning in a classical setting by techniques turning
around the notion of (modified) bar induction
\cite{berardi1998computational}, Krivine's
realizability~\cite{krivine2003dependent} and even more recently with
restriction on elimination of dependent sums and lazy
evaluation~\cite{herbelin2012constructive}.
%

Unfortunately, all those works---although they provide very nice and
sophisticated theoretical results---do not share the generality and
simplicity of sheafification in topos theory and are not amenable to a
direct integration into proof assistant based on type theory such as
Coq or Agda.  
%
However, type theory is known to be quite closed to topos theory so
one could wonder why similar techniques have not been developed in the
field of type theory. 
%
The answer to this question has been given recently by the advent of
homotopy type theory~\cite{hottbook}, which is an extension of
Martin-Löf type theory with principles inspired by (higher) category
theory and homotopy theory, such as higher inductive
types~\cite{lumsdaine2011higher,lumsdaine2013higher} and
Voevodsky's univalence principle~\cite{kapulkin2012simplicial}. 
%

This new point of view on type theory has revealed the
homotopy structure of types where for instance homotopical
propositions (or $\Type_{-1}$) are just types with an irrelevant equality and
homotopical sets (or $\Type_{0}$)  are types with a propositional
equality and so on for $\Type_{n}$.
%
The development of univalence has also shed some light on the
difficulty to make the AC and EM coexist in type theory. Indeed, it
has been shown that a naive (non-propositional) version of EM is
inconsistent with univalence.

When restricted to homotopical propositions and sets, homotopy type
theory corresponds quite closely to topos theory but the mismatch
starts when considering higher homotopical types.
%
Fortunately, a higher version of topos theory has been developed
recently by mathematicians, synthesized in the monograph of Lurie on
higher topos theory~\cite{lurie}. 
%
Lurie has developed all the tools that have been defined in
topos theory, but in a higher setting. In particular, the theory of
sheaves has been lifted to higher topos theory.
%
As the notion of higher topoi appears to correspond very closely
to homotopy type theory, this provides a new hope that tackling the problem of
extending the power of type theory using sheafification is actually possible.

Nevertheless, the adaptation of the sheafification in higher topos
theory to homotopy type theory is not straightforward because the
construction in higher topoi is restricted to the initial
Grothendieck which is still very topological, and not very amenable to
formalization in type theory. It seems more promising to use a
generalized sheafification, called Lawvere-Tierney
sheafification~\cite{maclanemoerdijk}, but this construction has not
been considered yet in the setting of higher topos theory. 

\subsection{Lawvere-Tierney Topology}

Lawvere-Tierney sheafification in a topos $\E$ is based on an abstract
point of view on the topology to be considered, being simply an
endomorphism on the classifying object $\Omega$ of $\E$  
%
$$
j : \Omega \rightarrow \Omega
$$
%
that is required to preserve $\True$ ($j \ \True = \True$), to be
idempotent ($j \circ j = j$) and compatible with products ($j \circ
\wedge = \wedge \circ (j, j)$).
%
A typical example is given by double negation.

Every Lawvere-Tierney topology $j$ induces a closure operator
$A \mapsto \closure{A}$ on subojects. If we see a subobject $A$ of $E$
as a characteristic function $\Char{A}$, the closure $\closure{A}$
corresponds to the subobject of $E$ whose characteristic function is 
%
$$
\Char{\closure{A}} = j \circ \Char{A}.
$$
%
A suboject $A$ of $E$ is said to
be dense when $\closure{A} = E$.

The idea is then to define sheaves in $\E$ as objects of $\E$ for
which it is impossible to make a distinction between objects and their
dense subobjects. This idea is formalized by saying that for every
dense subobject $A$ of $E$, the following canonical map is an
isomorphism
%
\begin{equation}\label{equ:sheaf_def}
\Hom{\E}(E,F) \rightarrow \Hom{\E}(A,F).
\end{equation}
%
On can show that $\Sh{\E}$, the full sub-category of $\E$ given by
sheaves, is again a topos, with classifying object
%
$$
\Omega_j = \{ P \in \E \ | \ j P  = P \}.
$$
%
Thus, in case of the double negation, the resulting topos is boolean  
and admits classical reasoning.

But furthermore, one can defined a left adjoint, the sheafification
functor, 
%
$$
a_j : \E \rightarrow \Sh{\E}
$$
to the inclusion functor, which exhibits $\Sh{\E}$ as a reflective
subcatefory of $\E$ (which is a particular case of localization). This
means that logical principles valid in $\E$ are still valid in
$\Sh{\E}$.


\subsection{Overview of the Result}

The main contribution of this article is the development of a
computer-checked definition of Lawvere-Tierney sheafification in the
setting of homotopy type theory.
%
This provides a way to give a meaning to the propositional law of
excluded middle, and more generally, combined with the previous work on
forcing in type theory~\cite{jaber2012extending}, it allows to reuse
results derived for topos theory, such as independence results.

To extend Lawvere-Tierney sheafification to homotopy type theory, the
first thing to realize is that the construction can not be done in one
step anymore. It must be done by induction on the level of homotopy of
types. More precisely, the first layer of sheafification is defined
for $\Type_{0}$ , given a topology on $\Type_{-1}$. This part
corresponds to sheafification for topoi. Then, assuming that the
sheafification has been constructed up to level $n$, one can defined
the sheafification for $\Type_{n+1}$.
%
This inductive process in the definition of sheafification is similar
to construction in the Grothendieck setting, where the
sheafification is obtained by applying the plus-construction $n+1$ to
$\Type_{n}$~\cite[Chapter~6]{lurie}.

This inductive step requires to formalize the notion of left-exact
reflective subuniverse, which corresponds to a stratified
version of left-exact modalities introduced
in~\cite[Chapter~7]{hottbook}. A Lawvere-Tierney topology can thus be
seen as a left-exact modality on $\Type_{-1}$ and sheafification as
a way to extend it as a left-exact modality on $\Type_{n}$ for any $n$.

Compare to the definition of sheafification in topos theory, the main
change comes from the fact that in homotopy type theory, there is not
a single object classifier but rather an $n$-object classifier for
each $n\geq -2$ which classifies functions with $n$-truncated homotopy
fibers~\cite{sets_in_hott}. This means that when monomorphisms are used in topos
theoretic definition, the correct generalization may be $n$-truncated homotopy
fibers or monomorphims (\ie $-1$-truncated homotopy
fibers) depending on the context.

For instance, the notion of separated objects (which is a technical
tool in the construction of the sheafification), for which the morphism
(\ref{equ:sheaf_def}) is just a monomorphism, is generalized in two
different ways.  
%
The subobjects to be considered are given by $n$-truncated fibers,
while the morphism (\ref{equ:sheaf_def}) is still required to be a
monomorphism.

In other technical difficulty is the fact that sheafification in topos
theory uses the fact that epimorphisms are coequalizers of their
kernel pairs. This properties does not holds in higher topoi, but it
can be replaced by Giraud's axiom which basically allows to replace 
 kernel pairs by \v{C}ech nerves.

Finally, one of the main difficulty in our formalization is that many
equlity which are trival (b yproof irrelevance) in the topos theoretic
setting becomes relevant and requires a carreful use of univalence
with appropriate isomorphisms.

\subsection{Plan of the Paper}

Section~\ref{sec:hott} defines the foundations of homotopy type theory
and the definition of modalities, on which this paper is based.
%
Section~\ref{sec:sheaves} defines the notion of sheaves in homotopy
type theory.
%
Section~\ref{sec:sheafification} depicts the construction of the
sheafification functor.
%
Section~\ref{sec:formalization} presents the details of the
  formalization in Coq.
%
Finally, Section~\ref{sec:future-works} concludes on future works.


\section{Preliminaries on Homotopy Type Theory}
\label{sec:hott}

In this section, we review some basic definitions in homotopy type
theory that are central in our formalization but not specific to
sheafification. 
%
Section~\ref{ssec:hott} introduces the notion of homotopy types and
classifying objects as defined in~\cite{sets_in_hott}.
%
Section~\ref{sec:colim-homot-type} introduces the notion of colimits
in homotopy type theory. 
%
And Section~\ref{sec:giraud-ax} defines (a particular case of)
Giraud's axiom on higher topoi that we need in the definition of
sheafification.

The definition of Section~\ref{ssec:hott} are part of the Coq library
available at \url{https://github.com/HoTT/HoTT}, while other definitions
and theorems are specific to our formalization. 


 \subsection{Homotopy Types and Classifying Objects}
\label{ssec:hott}

Our work is mainly based on the stratification by $\Type_n$~:
\begin{defi}
  $\IsType n$ is defined by induction on $n\geqslant -2$~:
  \begin{itemize}
  \item $\IsType {(-2)} X$ if $X$ is a contractible type, \ie $X$
    is pointed by $c:X$, and every other point in $X$ is connected to $c$.
  \item $\IsType {(n+1)} X = \prod_{x,y:X} \IsType n (x=y).$
  \end{itemize}
  Then, $\Type_n = \sum_{X:\Type} \IsType n X$.
\end{defi}
For a type $X$, to be in $\Type_n$ means that path spaces of $X$ are
trivial after $n+1$ iteration.
We define some shortcuts
\begin{itemize}
\item $\IsContr := \IsType {(-2)}$ and $\Contr := \Type_{-2}$
\item $\IsHProp := \IsType {(-1)}$ and $\HProp := \Type_{-1}$
\end{itemize}

Working in $\Type_n$ allows to use the subobjects classification as in~\cite{sets_in_hott}~: for
all type $B$, $\Type_n$ classifies subobjects of $B$ with $n$-truncated
fibers ; we have an equivalence
$$\left(\sum_{A:\Type} \sum_{f:A \to B} \prod_{b\in B}
\IsType n~\mathrm{fib}_f(b)\right) \simeq (B \to \Type_n)$$
and a pullback (with $\Type_n^\bullet$ the universe of pointed
$n$-truncated types)~:
$$
\xymatrix{
  A \ar[r]^{t_f} \ar[d]_f & \Type_n^\bullet \ar[d]^{\pi_1}\\
  B \ar[r]_{\chi_f} & \Type_n
}
$$
forall $f$ with $n$-truncated fibers, with 
$$t_f = \lambda a,~(\mathrm{fib}_f(f(a)),(a,\mathrm{idpath})).$$

Using higher inductive types, one can define for any type $X$ $\|X\| : \HProp$
generated by 
\begin{itemize}
\item a function $|\cdot|_X : X \to \|X\|$,
\item paths $x=y$ for all $x,y:\|X\|$.
\end{itemize}
$\|X\|$ is called the {\em (propositional) truncation of $X$}. The recursion principle asserts~:
\begin{lem}
  For any $A:\Type$ and $B:\HProp$, if $f:A \to B$ then there is an
  induced $g:\|A\|\to B$ such that $g(|a|)= f(a)$ for any $a:A$.
\end{lem}
We refer to~\cite{hottbook} for more details.

\subsection{Colimits in Homotopy Type Theory }
\label{sec:colim-homot-type}

\nt{Kevin -> complete this subsection}

Formalization of colimits is based on the formalization of limits in~\cite{lumsdaine}.

Definition of graphs, diagrams, is\_colimit. 

Maybe the HIT of colimit, at least say that using HIT, all colimits
exists. 

\subsection{Giraud's axiom}
\label{sec:giraud-ax}

\begin{defi}
  Let $f:X \to Y$ be a map, and $p:\nat$. The $p$-pullback of $f$,
  noted $X\times_Y \cdots\times_Y X$ is
  the limit of the diagram
  $$\xymatrix{
    X \ar[rd]^f & \cdots & X \ar[ld]_f \\
      &    Y   &
  }$$
  with $p$ copies of $X$. The $0$-pullback of $f$ is $\Unit$.
\end{defi}

In homotopy type theory, we have 
$$X\times_Y \cdots\times_Y X = \sum_{x:X^n} (f x_1 = f x_2) \land
\cdots \land (f x_{n-1} = f x_n).$$
\begin{defi}
  If $f:X \to Y$ is a map, then the {\em \v{C}ech nerve} of $f$ is the
  diagram
  $$\xymatrix{
    C(f) := \cdots~ X \times_Y X \times_Y X \tar[r]{4pt} & X \times_Y X \dar[r]{2pt} & X
  }$$
where the arrows are the canonical projections.
\end{defi}

\begin{ax}[Giraud]
  If $f:X\to Y$ is a surjection, then the
colimit of $C(f)$ is $Y$.
\end{ax}

Viewing homotopy type theory as a
type-theoristic version of higher topos theory, \cite{lurie} suggests that this
axiom can be proved, although we admit it in this paper.



\section{Left exact Modalities}
\label{sec:lexmod}

\cite{shulman-higher-modalities}

\subsection{Definition}

We use the following definition of truncated left exact modalities~:
\begin{defi}
  Let $p$ be a truncation index. Then a left exact modality is the
  data of
  \begin{enumerate}[(i)]
  \item A predicate $P:\Type_p \to \HProp$
  \item For every $p$-truncated type $A$, a $p$-truncated type
    $\modal A$ such that $P(\modal A)$
  \item For every $p$-truncated type $A$, a map $\eta_A:A \to
    \modal A$
  \end{enumerate}
  such that
  \begin{enumerate}[(i)]
    \setcounter{enumi}{3}
  \item For every $p$-truncated types $A$ and $B$, if $P(B)$ then
    $$\left\{
      \begin{array}{rcl}
        (\modal A \to B) & \to & (A \to B) \\
        f & \mapsto & f \circ \eta_A
      \end{array} \right.$$
    is an equivalence.
  \item For all $A:\Type_p$ and $B:A \to \Type_p$ such that $P(A)$
    and $\prod_{x:A} P(B x)$, then $P\left( \sum_{x:A} B(x)\right)$
  \item For all $A:\Type_p$ and $x,y:A$, if $\modal A$ is
    contractible, then $\modal (x=y)$ is contractible.
  \end{enumerate}
  Conditions (i) to (iv) define a {\em reflective subuniverse}, (i) to
  (v) a {\em modality}.
\end{defi}

The type of all $p$-types such that $P$ will be noted $\Type^\modal_p$

Since basic operations (dependent products, products, sigma types) are
stable for truncation levels, all theorem in~\cite{hottbook}, chapter
7.7 remains true.
In particular, if $X:\Type_p$ verifies $P(X)$, then for any $x,y:X$, $P(x=y)$.

Another useful property is that we can extend property {\it (iii)} to
dependent product.

Moreover, left exactness implies in particular fibers preservations~:
\begin{prop}
For any $n$-truncated types $X$ and $Y$,
and any map $f:X \to Y$, the modalisation of fiber of $f$ above any element $y:Y$
is the fiber of $\modal f$ above $\eta_Y y$~:
$$\modal \left(\sum_{x:X}  (f x = y)\right) = \sum_{x:\modal X} (\modal f x = \eta_Y y).$$

Moreover the following diagram commute
$$\xymatrix{
  \sum_{x:X} (f x = y) \ar[r]^\eta \ar[d]_\gamma & \modal \left(\sum_{x:X}  (f x = y)\right) \ar@{=}[dl]\\
  \sum_{x:\modal X} (\modal f x = \eta_Y y) &
}$$
where $\pi_1 \circ \gamma = \eta_X$, and $\pi_2 \circ \gamma$ is the
usual modalisation of paths.
\end{prop}


As we want, a left exact modality preserves $\HProp$~:
\begin{prop}
  If $P:\HProp$, then $\modal \widehat P : \HProp$, where $\widehat P$
  is $P$ seen as a $p$-type.
\end{prop}

\subsection{Examples of Left Exact Modalities}

A basic example is the {\em open modality of $P$} for a $P:\HProp$,
$\modal_P T = P \to T$.  When one works in the subuniverse defined by
this modality, defining an inhabitant of a type $T$ is actually giving
an inhabitant of $P \to T$. Thus, being in this subuniverse is just
adding the axiom $P$ to the theory. Sadly, adding an axiom this way
does not add any computational content to $P$, thus is not completely
satisfactory.

Another example is the $\lnot\lnot$-modality, $\modal_{\lnot\lnot} T = (T
\to \bot) \to \bot$. Although we might think it enables us to use
classical reasonings in the corresponding reflective subuniverse, the
main problem is that every type in the new universe is an $\HProp$,
which is very restrictive. 
We will use this modality only in the $\HProp$ layer, and try to
extend it to every $\Type_n$ without collapsing every type.


\subsection{The new type theory arising from a left exact modality}

A left exact modality leads to a new homotopy type theory with
potentially new principles. 

Definition of Sigma-Types. 

Preservation of univalence.

Consistency issues.


\section{Sheaves}
\label{sec:sheaves}

We will define sheaves by induction on the homotopical truncation level of types
~: the base case will be any left exact modality on $\HProp$, and we
give in this section some definitions useful for the inductive case.

% \subsection{Definitions and first properties}
% \label{sec:def}

In this section, we suppose given a truncation index $n\geqslant -1$,
and a left exact modality $\modal$ on $\Type_n$.

\begin{defi}
  Let $E$ be a type. The {\em closure} of a subobject of $E$ classified by $\chi$
  is the subobject of $E$ classified by $\modal \circ \chi$.

  The subobject of $E$ classified by $\chi$ is said {\em closed in
    $E$} if its closure is itself : $\chi = \modal \circ \chi$.
\end{defi}

\begin{defi}
  Let $E$ be a type, and $\chi:E \to \Type_n$. The subobject $A$ of $E$
  classified by $\chi$ is {\em dense} in $E$ if its $\modal$-closure
  is $E$ seen as a subobject of $E$, \ie
  $$\forall e:E,~ \left(\sum_{e':E} e=e'\right) \simeq (\modal (\chi~e)).$$ 
  Moreover, for all $x:A$ we need the following coherence diagram to
  commute
  $$\xymatrix{
    \sum_{e':A} x = e' \ar@{=}[r] \ar[d]_\iota & (\chi~x) \ar[d]^{\eta_{(\chi~x)}}\\
    \sum_{e':E} x = e' \ar@{=}[r] & \modal (\chi~x)
  }$$
  where $\iota: x \mapsto ({x_1}_1 ; x_2)$.
\end{defi}

It follows from fibers preservation that any $n$-subobject of a type seen as a $n$-subobject
of its closure is closed.

\begin{defi}[Restriction]
  For any $A, E:\Type$ and $F:\Type_{n+1}$, we define~:
  \begin{itemize}
  \item if $A \hookrightarrow E$, $M_E^\chi$ is the map sending an
    arrow $f~:~E\to F$ to $\restriction f A$ the restriction of $f$ to $A$.
  \item if $A \to E$ with $n$-truncated fibers, $\Phi_E^\chi$ is the
    map sending $f~:~E \to f$ to $\restriction f A$.
  \end{itemize}
\end{defi}

\begin{defi}[Separated Type]
  A type $F$ in $\Type_{n+1}$ is {\em separated} if for all type $E$, and
  all dense subobject of $E$ classified by $\chi~:~E \to \Type_n$,
  $\Phi_E^\chi$ is a monomorphism. In other words, the dotted arrow,
  if exists, is unique.

  $$\xymatrix{
    \sum_{e:E} (\chi~ e) \ar[r]^f \ar[d]_{\pi_1} & F \\
    E \ar@{-->}[ru]_{!}&
  }$$
\end{defi}

\begin{defi}[Sheaves]
  A type $F$ of $\Type_{n+1}$ is a {\em $(n+1)$-sheaf} if it is
  separated, and for all type $E$ and all dense subobject of $E$
  classified by $\chi~:~E \to \Type_{-1}$, $M_E^\chi$ is an
  equivalence. In other words, the dotted arrow exists and is unique.

  $$\xymatrix{
    \sum_{e:E} (\chi~ e) \ar[r]^f \ar[d]_{\pi_1} & F \\
    E \ar@{-->}[ru]_{\exists !}&
  }$$
\end{defi}

Note that these definitions are almost the same as the ones
in~\cite{maclanemoerdijk}. The main difference is that {\em separated}
is defined for $n$-subobjects,
while {\em sheaf} only for $-1$-subobject.

From these definitions, one can show that
\begin{prop}\label{prop:sheaf_prop}
  \begin{itemize}
  \item $\Type_n^\modal$ is a sheaf.
  \item If $A:\Type_{n+1}$ and $B:A \to \Type_{n+1}$ such that for all
    $a:A$, $(B~a)$ is a sheaf, then $\prod_{a:A}B~a$ is a sheaf.
  \end{itemize}
\end{prop}

\section{Construction of sheafification}
\label{sec:sheafification}

We mimic the construction in~\cite{maclanemoerdijk}. 
From any left exact modality on $\HProp$, we extend the new principles
it gives to every $\Type_n,\,n\geqslant 0$.
Actually, if $n_0$ is a fixed truncation index, and $\modal$ a left
exact modality on $\Type_{n_0}$, then we can in the same
way extend the properties of the modality to any $\Type_n$, for
$n > n_0$.

Once the first modality is defined, the extension to all $\Type_n$ is
automatic. It means that the new principles we want to add in the new
theory must be introduced ine the first modality. 

We have to be careful~: if we want to propagate the properties of the
first modality $\modal_{-1}$, all the higher modalities $\modal_n$ must be compatible
with $\modal_{-1}$, \ie we want the property
\begin{prop}\label{prop:hprop}
  If $P:\HProp$, and $\widehat P$ is $P$ seen as a $\Type_n$, then $\modal_n \widehat P = \modal_{-1} P$, and the
  following coherence diagram commute 
  $$\xymatrix{
    \modal_{-1} P \ar@{=}[r] & \modal_n \widehat P \\
    P \ar@{=}[r] \ar[u]^{\eta_{-1}} & \widehat P \ar[u]_{\eta_n}
  }$$
\end{prop}

The sheafification process works exactly that way
in~\cite{maclanemoerdijk}~: from a left exact modality on the internal
logic (a Lawvere-Tierney topology), we define a new left exact
modality on the whole topos. We view that as the induction step
between $\HProp$ and $\Type_0$, and we want to extend it.

\kq{Bof ; maybe rewrite this}

As in most of sheafification process, it will be done in
two steps~:
\begin{enumerate}[(i)]
\item {\em separation~:} In this step, we {\em remove} the lack of
  sheafness from the type we are considering.
\item {\em completion~:} In this step, we add what lacks now to the
  separated type to be a sheaf.
\end{enumerate}
\subsection{For h-propositions}
\label{ssec:h-propositions}

For the case $n=-1$, one can take any left exact modality on
$\HProp$. Here we took the $\lnot\lnot$ modality~:
$$\forall P:\HProp,~\modal_j P = \lnot\lnot P.$$
One can easily show that it defines a left exact modality.

The $\lnot\lnot$-modality allows to work with a propositional law of
excluded middle. 

\subsection{From Type to Separated Type}
\label{ssec:from-type-separated}
In this section, we suppose given a truncation index $n\geqslant -1$,
and a left exact modality $\modal$ on $\Type_n$, compatible with the
modality on $\HProp$ in the sense of proposition~\ref{prop:hprop}.

Let $T$ be in $\Type_{n+1}$. We define $\square T$ as the image of
$\modal^T \circ \{\cdot\}_T$, as in
$$\xymatrix{
  T \ar[r]^{\{\cdot\}_T} \ar[d]_{\mu_T} & \Type_n^T \ar[d]^{\modal^T} \\
  \square T \ar[r]& \left( \Type_n^\modal \right)^T
}, $$
where $\{\cdot\}_T$ is the singleton map $\lambda (t:T),~\lambda
(t':T),~t=t'$.
In type theory words, 
\begin{align*}
\square T &= \im (\lambda~t:T,~\lambda~ t',~ \modal (t = t')) \\
          &= \sum_{u:T \to \Type_n^\modal} \left\| \sum_{a:X} 
            (\lambda t,~\modal (a=t)) = u\right\|
\end{align*}

There again, the separation step has the same definition as
in~\cite{maclanemoerdijk}, using $\Type_n^\modal$ instead of the
$j$-subobject classifier.

\begin{prop}
  For all $T:\Type_{n+1}$, $\square T$ is separated.  
\end{prop}

\begin{proof}
We will use the following lemma~:
\begin{lem}
  A $(n+1)$-truncated type $T$ with an embedding $f : T \to U$
  into a separated $(n+1)$-truncated type $U$ is itself separated.
\end{lem}
As $\square T$ embeds in $\left( \Type_n^\modal \right)^T$, we only
have to show that the latter is separated. But it is the case with
both parts proposition~\ref{prop:sheaf_prop}.
\end{proof}

We now need to show that $\square$ is universal. The following
lemma is central in the proof~:
\begin{lem}\label{lem:sepiscolim}
  Let $T:\Type_{n+1}$. Then $\square T$ is the colimit of the closed
  diagonal diagram
  $$\xymatrix @C=4em  { 
    \cdots \overline{\Delta_3} \tar[r]{4pt} & \overline{\Delta_2} \dar[r]{2pt} &
    \overline{\Delta_1}
  }$$
where $\Delta_k$ is the $k$-pullback of $\id : T \to T$.
\end{lem}

This lemma is an adaptation of the sheafification process
in~\cite{maclanemoerdijk}, where they consider only the kernel pair of
$\mu_T$ instead of the \v{C}ech nerve.

\begin{proof}
  As $\mu_T$ is an surjection, we know by Giraud axiom that $\square T$ is the colimit
  of $C(\mu_T)$. If we can show that $C(\mu_T) = C(\id)$, the the
  result will follow. 

  Let $k:\nat$, let's show that 
  \begin{align*}
    &\sum_{t:T^k} (\mu_T t_1 = \mu_T t_2 \land \cdots
      \land \mu_T t_{k-1} \mu_T t_k)\\
    &= \sum_{t:T^k} \modal (t_1 = t_2 \land \cdots
      \land t_{k-1} = t_k)
  \end{align*}
By induction on $k$, and the preservation of products by $\modal$, it
suffices to show that for all $a,b:T$, $\modal (a=b) = (\mu_T a =
\mu_T b)$. By univalence, we want arrows in both ways, forming an
equivalence.
\begin{itemize}
\item Suppose $p : (\mu_T a = \mu_T b)$. Then projecting $p$ along
  first components yields $q : \prod_{t:T} \modal(a=t) = \modal (b=t)
  $.
  Taking for example $t=b$, we deduce $\modal (a=b) = \modal(a=a)$,
  and the latter is inhabited by $\eta_{a=a} 1$.
\item Suppose now $p : \modal(a=b)$. Let $\iota$ be the first
  projection from $\square T \to (T \to \Type_n^\modal)$. $\iota$ is
  an embedding, thus it suffices to prove $\iota (\mu_T a) = \iota
  (\mu_T b)$, \ie $\prod_{t:T}\modal (a=t) = \modal (b=t)$. The latter
  remains true by univalence.
\end{itemize}
The fact that these two form an equivalence is technical, we refer to
the formalization for an explicit proof.
\end{proof}

Now, let $Q$ be a separated $\Type_{n+1}$, and $f:T \to Q$. Then the
following diagram commutes
$$\xymatrix @C=4em{ 
    \cdots \overline{\Delta_3} \tar[r]{4pt} \ar[rd]_{f\circ \pi_1} & \overline{\Delta_2}
    \dar[r]{2pt} \ar[d]^{f\circ \pi_1} &
    \overline{\Delta_1} = T \ar[ld]^{f\circ \pi_1}\\
    & Q &
  }$$
But we know (lemma~\ref{lem:sepiscolim}) that $\square T$ is the
colimit of the closed diagonals diagram, thus there is an universal
arrow $\square T \to Q$.

This proves the following proposition
\begin{prop}
  $(\square,\mu)$ defines a reflective subuniverse on $\Type_{n+1}$.
\end{prop}

We now have to prove point {\it (v)} in the definition of
modality. Let $A:\Type_{n+1}$ be a sheaf and $B:A \to \Type_{n+1}$ be
a sheaf family. We want to show that $\sum_{x:A} (Bx)$ is
separated. Let $E$ be a type, and $\sum_{e:E} (\chi\,e)$ a dense
subobject of E.

Let $f,g$ be two maps from $\sum_{e:E} (\chi\,e)$ to $\sum_{x:A}
(Bx)$, equal when precomposed with $\pi_1$.
$$\xymatrix @R=4em @C=4em{
  \sum_{e:E} (\chi\, e) \ar@<-2pt>[r]_{g\circ\pi_1} \ar@<2pt>[r]^{f\circ \pi_1} \ar[d]_{\mathrm{dense}}& \sum_{x :A} (Bx) \\
  E \ar@<-2pt>[ru]_{g} \ar@<2pt>[ru]^{f}&
}$$
We can restrict the previous diagram to 
$$\xymatrix @R=4em @C=5em{
  \sum_{e:E} (\chi\, e) \ar@<-2pt>[r]_{\pi_1\circ g\circ\pi_1} \ar@<2pt>[r]^{\pi_1\circ f\circ \pi_1} \ar[d]_{\mathrm{dense}}& \sum_{x :A} (Bx) \\
  E \ar@<-2pt>[ru]_{\pi_1\circ g} \ar@<2pt>[ru]^{\pi_1\circ f}&
}$$
and as $A$ is separated, $\pi_1\circ f = \pi_2 \circ g$.
For the second components, let $x:E$. Notice that for all $x:E$,
$\sum_{y:E} x = y$ has a dense subobject, $\sum_{y:\sum_{e:E} (\chi\,
  e)} x=y_1$~:

$$\xymatrix{
  \sum_{y:\sum_{e:E} (\chi\,
  e)} x=y_1 \dar[rrr]{2pt}^{\pi_2\circ f\circ\pi_1\circ \pi_1}_{\pi_2\circ g\circ \pi_1\circ \pi_1} \ar[d]_{\mathrm{dense}}&&& B\,x \\
  \sum_{y:E} x = y \dar[rrru]{2pt}^{\pi_2\circ f\circ \pi_1}_{\pi_2\circ g\circ \pi_1}&
}$$
Using the sepatation property of $B\,x$, one can show that second
components, transported correctly along the first components equality,
are equal. The complete proof can be found in the formalisation.
This proves the following proposition
\begin{prop}
  $(\square,\mu)$ defines a modality on $\Type_{n+1}$.
\end{prop}

As this modality is just a step in the construction, we will not show
that it is left exact, we will do it only for the sheafification modality.

\subsection{From Separated Type to Sheaf}
\label{ssec:separated-to-sheaf}

If $T$ is already a separated type, the following lemma alllows us
to build a sheaf~:
\begin{lem}
  Let $T:\Type_{n+1}$ be separated, and $U$ be a sheaf. If $T$ embeds
  in $U$, and is closed in $U$, then $T$ is a sheaf.
\end{lem}

As any separated type $T$ embeds in $\left(\Type_n^\modal\right)^T$,
it suffices to take the closure of $T$ to get a sheaf, and the unit
$\nu_T:T \to \star T$ is obvious.
\begin{prop}
  $(\star,\nu)$ defines a reflective subuniverse.
\end{prop}
\begin{proof}
  Let $T,Q:\Type_{n+1}$ such that $Q$ is a sheaf. Let $f:T\to Q$.
  Because $Q$ is a sheaf, it is in particular separated ; thus we can
  extend $f$ to $\square f:\square T\to Q$.

  But as $\star T$ is the closure of $\square T$, $\square T$ is dense
  into $\star T$, so the sheaf property of $Q$ allows to extend
  $\square f$ to $\star f:\star T \to Q$.

  As all these steps are universal, the composition is.
\end{proof}


\subsection{Summary}
\label{ssec:summary}

If $T:\Type_{n+1}$, the sheafification of $T$ is
$$\star T = \sum_{u:T \to \Type_n^\modal} \lnot\lnot\left\| \sum_{a:X} 
            (\lambda t,~\modal (a=t)) = u\right\|$$

We already know that $\star$ defines a modality. It remains to show
that this modality is left exact.

The last thing to prove is the cumulativity of our sheafifications~:
\begin{prop}
  If $T:\Type_n$, then $\modal T = \star \widehat T$ (where $\widehat
  T$ is $T$ seen as a $\Type_{n+1}$.
\end{prop}
It will imply that $\star$ is compatible with $\HProp$ in the sense of proposition~\ref{prop:hprop}.



% An example of a floating figure using the graphicx package.
% Note that \label must occur AFTER (or within) \caption.
% For figures, \caption should occur after the \includegraphics.
% Note that IEEEtran v1.7 and later has special internal code that
% is designed to preserve the operation of \label within \caption
% even when the captionsoff option is in effect. However, because
% of issues like this, it may be the safest practice to put all your
% \label just after \caption rather than within \caption{}.
%
% Reminder: the "draftcls" or "draftclsnofoot", not "draft", class
% option should be used if it is desired that the figures are to be
% displayed while in draft mode.
%
%\begin{figure}[!t]
%\centering
%\includegraphics[width=2.5in]{myfigure}
% where an .eps filename suffix will be assumed under latex, 
% and a .pdf suffix will be assumed for pdflatex; or what has been declared
% via \DeclareGraphicsExtensions.
%\caption{Simulation results for the network.}
%\label{fig_sim}
%\end{figure}

% Note that IEEE typically puts floats only at the top, even when this
% results in a large percentage of a column being occupied by floats.


% An example of a double column floating figure using two subfigures.
% (The subfig.sty package must be loaded for this to work.)
% The subfigure \label commands are set within each subfloat command,
% and the \label for the overall figure must come after \caption.
% \hfil is used as a separator to get equal spacing.
% Watch out that the combined width of all the subfigures on a 
% line do not exceed the text width or a line break will occur.
%
%\begin{figure*}[!t]
%\centering
%\subfloat[Case I]{\includegraphics[width=2.5in]{box}%
%\label{fig_first_case}}
%\hfil
%\subfloat[Case II]{\includegraphics[width=2.5in]{box}%
%\label{fig_second_case}}
%\caption{Simulation results for the network.}
%\label{fig_sim}
%\end{figure*}
%
% Note that often IEEE papers with subfigures do not employ subfigure
% captions (using the optional argument to \subfloat[]), but instead will
% reference/describe all of them (a), (b), etc., within the main caption.
% Be aware that for subfig.sty to generate the (a), (b), etc., subfigure
% labels, the optional argument to \subfloat must be present. If a
% subcaption is not desired, just leave its contents blank,
% e.g., \subfloat[].


% An example of a floating table. Note that, for IEEE style tables, the
% \caption command should come BEFORE the table and, given that table
% captions serve much like titles, are usually capitalized except for words
% such as a, an, and, as, at, but, by, for, in, nor, of, on, or, the, to
% and up, which are usually not capitalized unless they are the first or
% last word of the caption. Table text will default to \footnotesize as
% IEEE normally uses this smaller font for tables.
% The \label must come after \caption as always.
%
%\begin{table}[!t]
%% increase table row spacing, adjust to taste
%\renewcommand{\arraystretch}{1.3}
% if using array.sty, it might be a good idea to tweak the value of
% \extrarowheight as needed to properly center the text within the cells
%\caption{An Example of a Table}
%\label{table_example}
%\centering
%% Some packages, such as MDW tools, offer better commands for making tables
%% than the plain LaTeX2e tabular which is used here.
%\begin{tabular}{|c||c|}
%\hline
%One & Two\\
%\hline
%Three & Four\\
%\hline
%\end{tabular}
%\end{table}


% Note that the IEEE does not put floats in the very first column
% - or typically anywhere on the first page for that matter. Also,
% in-text middle ("here") positioning is typically not used, but it
% is allowed and encouraged for Computer Society conferences (but
% not Computer Society journals). Most IEEE journals/conferences use
% top floats exclusively. 
% Note that, LaTeX2e, unlike IEEE journals/conferences, places
% footnotes above bottom floats. This can be corrected via the
% \fnbelowfloat command of the stfloats package.

\section{Formalization} 
\label{sec:formalization}

A Coq formalization is available at
\url{https://github.com/KevinQuirin/sheafification/}.

\nt{describe how many line for each, be clear about what has not been formalized}

Talk about issues in universe levels. Lake of cumulativity for
$\Type_n$ and impossibility to define by induction the sheafification 
because universe levels must be the same for every $n$.

\section{Conclusion and Future Works}
\label{sec:future-works}

Prove that the Giraud axiom holds in homotopy type theory.

Extend the sheafification to all types, even if non truncated.





% conference papers do not normally have an appendix


% use section* for acknowledgment
\section*{Acknowledgment}


The authors would like to thank...


% trigger a \newpage just before the given reference
% number - used to balance the columns on the last page
% adjust value as needed - may need to be readjusted if
% the document is modified later
%\IEEEtriggeratref{8}
% The "triggered" command can be changed if desired:
%\IEEEtriggercmd{\enlargethispage{-5in}}

% references section

% can use a bibliography generated by BibTeX as a .bbl file
% BibTeX documentation can be easily obtained at:
% http://www.ctan.org/tex-archive/biblio/bibtex/contrib/doc/
% The IEEEtran BibTeX style support page is at:
% http://www.michaelshell.org/tex/ieeetran/bibtex/
\bibliographystyle{IEEEtran}
\bibliography{sheaf_lics}
% argument is your BibTeX string definitions and bibliography database(s)
%\bibliography{IEEEabrv,../bib/paper}
%
% <OR> manually copy in the resultant .bbl file
% set second argument of \begin to the number of references
% (used to reserve space for the reference number labels box)
% \begin{thebibliography}{5}

% \bibitem[RS13]{sets_in_hott}
% Egbert Rijke and Bas Spitters.
% \newblock Sets in homotopy type theory.
% \newblock 2013.

% \bibitem[{Uni}13]{hottbook}
% {Univalent Foundations Project}.
% \newblock {\em Homotopy Type Theory: Univalent Foundations for Mathematics}.
% \newblock 2013.

% \bibitem[MM92]{maclanemoerdijk}
% Saunders MacLane and Ieke Moerdijk.
% \newblock {\em Sheaves in Geometry and Logic}.
% \newblock Springer-Verlag, 1992.

% \bibitem[HTT]{lurie}
% Jacob Lurie.
% \newblock{\em Higher Topos Theory}.

% \bibitem[AKL]{lumsdaine}
% Peter LeFanu Lumsdaine, Jeremy Avigad and Chris Kapulkin.
% \newblock{\em Homotopy limits in type theory}.

% \end{thebibliography}




% that's all folks

\end{document}
